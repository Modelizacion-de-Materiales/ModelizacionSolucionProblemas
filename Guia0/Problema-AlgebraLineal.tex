\section{Problema de Algebra Lineal}

\mode<article>

Aquí tenemos un muy simple problema de álgebra lineal. 
El objetivo del mismo es simplemente introducir las herramientas
que nos permitiran resolverlo. De nunguna manera se 
prentende que el alumno programe la solución . 

Nuestro problema es sencillo, 

\mode*

\begin{frame}[label=FrameEquationAlgebraLineal]
  \frametitle<presentation>{Problema de Algebra Lineal}

  \begin{equation}
    \left\{
      \begin{aligned}
	x - 3y -z &= 6 \\
	2 x -4 y -3 z &= 8 \\
	-3 x + 6 y + 8z &= -5
      \end{aligned}
    \right.
  \end{equation}
  Matricialmente:
  \begin{equation}
    \underbrace{
    \begin{pmatrix}
      1 & -3 & -1 \\
      2 & -4 & -3 \\
      -3 & 6 & 8 
    \end{pmatrix}
  }_A
    \underbrace{
    \begin{pmatrix}
      x \\ y \\ x
    \end{pmatrix}
  }_X
    =
    \underbrace{
    \begin{pmatrix}
      6 \\8 \\ -5
    \end{pmatrix}
  }_B
  \end{equation}

  La solución es inmediata:

  \begin{equation}
    X = A^{-1} B
  \end{equation}
\end{frame}

\mode<article>

La implementación implica usar las funciones disponibles
en nuestras herramientas. Como vimos en el apunte 
de \emph{introducción a la programación}, disponemos de
la funcion \texttt{numpy.linalg.solve} en \emph{Python}
o del operador barra invertida  ( \texttt{\textbackslash} ) en \emph{matlab}.

\mode*
\mode<all>
