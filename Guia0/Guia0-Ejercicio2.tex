\section{Expansión en serie, tolerancia y error}

\mode<article>

En este ejercicio pretenemos conocer qué tan bien
podemos aproximarnos al valor de la función exponencial 
con la expansión en serie alrededor de cero.
La \autoref{EquationExponentialnError} define tanto la 
serie hasta su \texttt{n-esimo} término como el error
cometido por la serie truncada. 

\mode*

\begin{frame}[label=FrameEquationExponencial]
  \frametitle<presentation>{Aproximación a la exponencial}

  \begin{equation}\label{EquationExponentialnError}
    \begin{split}
      \exp{x} = \sum_{i=0} ^n \frac{x^i}{i!}+ ERR[n] \\
	ERR [ n ] = \Bigl \vert 
		\frac {exp(x_o) - \sum\limits_{i=0} ^n \frac{x_o^i}{i!} } { exp(x_o) } \Bigr \vert
    \end{split}
  \end{equation}

\end{frame}

\mode<article>

El valor de \texttt{n} indica la cantidad de términos que tomamos en 
nuestra expansión, la cual dependerá de que tan ``tolerantes''
seamos con nuestro programa. Por ejemplo, una ``tolerancia'' 
de $1 \cdot 10^{-4} $. En el código de la \autoref{FigFrameShowCodeSerie} 
se observa cómo se va incrementando el valor de \texttt{n}
hasta que el error es menor que dicha tolerancia. Si bien esta 
medida no debe hacerse cada vez que se resuelve el problema, 
sí es necesario indicar el último término de truncamiento 
y una estimación del error cometido. 

\begin{figure}
  \includeslide[width=\textwidth]{FrameShowCodeSerie}
  \caption{\protect\label{FigFrameShowCodeSerie}
  Implementación en Python de la serie truncada 
  y el cáclulo del error cometido respecto de 
  $x_o = 0.5 $}
\end{figure}
\mode*

\begin{frame}<presentation>[label=FrameShowCodeSerie]
  \frametitle{Implementación de una expansion en serie truncada}

  \lstinputlisting[language=Python, lastline=16]{./Guia0_python/Ejercicio2/Ejercicio2.py} 

\end{frame}

\begin{frame}<presentation>[label=FrameComportamientoErrorvsN]
  \frametitle{Comportamiento del error con N}
  \center
%  \begin{figure}
    \includegraphics[height=0.8\textheight]{Guia0_python/Ejercicio2/Figura2.pdf}
%  \end{figure}

\end{frame}

\mode<all>
