\section{Ecuaciones Diferenciales Ordinarias}
\mode<article>

\begin{figure}
  \includeslide[width=\textwidth]{FrameEDO1}

  \caption{\protect\label{FiguraEDO1} Para resolver el problema necesitamos escribir 
  la ecuación diferencial a resolver de manera de reconocer la forma
  $\partial y / \partial x = F(x,y) $ }

\end{figure}

\begin{figure}
  \includeslide[width=\textwidth]{FrameEDO2-Metodos}
  \caption{\protect\label{FiguraEDO2-Metodos} 
  Los distintos métodos realizan aproximaciones sucesivas al paso siguiente 
  en función del paso acutal. Al aumentar la precision de la aproximación 
  se necesitan varias evaluaciones de la función. 
  }
\end{figure}
\begin{figure}

  \includeslide[width=\textwidth]{FrameEDO3-Errores}
  \caption{\protect\label{FiguraEDO3-Errores} 
  Como se ve aquí, a pesar de necesitar mas evaluaciones de la función, el método de Runge
  Kutta de orden 4 alcanza precisiones mucho mas altas que los otros métodos. 
  }

\end{figure}
\mode*

\begin{frame}<presentation>[label=FrameEDO1]
  \frametitle{Ecuaciones Diferenciales Ordinarias}
\center
  \includegraphics[width=0.8\textwidth,page=11,trim=0cm 0cm 0cm 4.2cm,clip]{./Resumen/GUIA1-Resumen-2018.pdf}

\end{frame}

\begin{frame}<presentation>[label=FrameEDO2-Metodos]
  \frametitle{EDO - Métodos de resolución}
\center
  \includegraphics[width=0.8\textwidth,page=12,trim=0cm 0cm 0cm 4.2cm,clip]{./Resumen/GUIA1-Resumen-2018.pdf}
\end{frame}

\begin{frame}<presentation>[label=FrameEDO3-Errores]
  \frametitle{EDO - Errores}
\center
  \includegraphics[width=0.8\textwidth,page=13,trim=0cm 0cm 0cm 4.2cm,clip]{./Resumen/GUIA1-Resumen-2018.pdf}
\end{frame}
\mode<all>
