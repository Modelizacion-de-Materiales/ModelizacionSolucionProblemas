\section{Splines}

\mode<article>

Este problema ya ha sido tratado en el apunte ``Cuentas 
de Interpolación''. Aquí usaremos los resultados que habíamos obtenido allí.
simplemente recordaremos que tenemos una serie de puntos experimentales 
por lo que queremos hacer pasar una función definida a tramos. 

\mode*
\begin{frame}<presentation>[label=FrameDefinicionSplines]
  \frametitle{Splines}
  \begin{equation}
    \begin{split}
    { (X_i, Y_i) }_{i = 1 \dots N}\\
      I_i = \left[ X_i , X_{i+1} \right] \\
      f_i = a_i (x - X_i)^3 + b_i (x-X_i)^2 + c_i (x-X_i) + d_i
    \end{split}
  \end{equation}
\end{frame}

\mode<article>
En cada intervalo $I_i$ vale el polinomio $f_i$ de grado 3. A estos polinomios
se les pide que sean continuos y que la primer y segunda derivada. Eso 
desemboca en un sistema de ecuaciones que termina en un sistema lineal para
los coeficientes $b_i$ . 

\mode*
\begin{frame}[label=FrameSistemaEcuacionesSplines]
  \frametitle<presentation>{Sistema De Ecuaciones para Splines}

  %  \includegraphics[width=\textwidth,page=4]{Resumen/GUIA1-Resumen-2018.pdf}
  \tiny
  \begin{equation}
   \begin{split}
      \begin{pmatrix}
	1 &     0      &    0    & 0 & \dots & 0 & 0 & 0 & 0 \\
	h_1& 2(h_2+h_1)& h_2     & 0 & \dots & 0 & 0 & 0 & 0 \\
	0  &  h_2  & 2(h_3+h_2) &h_3 & \dots & 0 & 0 & 0 & 0 \\
	\hdotsfor{9}\\
	0 & 0 & 0 & 0 & \cdots & h_{N-3} & 2(h_{N-2}+h_{N-3}) & h_{N-2} & 0 \\
	0 & 0 & 0 & 0 & \cdots & 0 & h_{N-2} &2(h_{N-1}+h_{N-2}) & h_{N-1} \\
	0 &     0      &    0    & 0 & \dots & 0 & 0 & 0 & 1
      \end{pmatrix} 
       \\
      \times 
      \begin{pmatrix}
	b_1 \\ b_2 \\ b_3 \\ \vdots \\ b_{N-2} \\ b_{N-1} \\ b_{N}
      \end{pmatrix} 
      = 
      3
      \begin{pmatrix}
	0 \\ 
	\dfrac{ Y_3 - Y_2 }{h_2} - \frac{ Y_2 - Y_1 }{h_1} \\
	\vdots \\
	\dfrac{ Y_N - Y_{N-1} }{h_{N-1}} - \frac{ Y_{N-1} - Y_{N-2} }{h_{N-2}} \\
	0
      \end{pmatrix}
    \end{split}
  \end{equation}
\end{frame}

\mode<article> 

Luego, al resolver el sistema de ecuaciones que sale de las condiciones
de borde para cada $f_i$ , tenemos las soluciones para $a_i$ y $c_i$ a partir de los 
$b_i$

\mode*

\begin{frame}[label=FrameEquationRecurrencias]
  \frametitle<presentation>{Coeficientes $a_i$ y $c_i$}
  \begin{equation}
    \begin{aligned}
      d_i =& Y_i \\
      a_i =& \frac{1}{3} \dfrac{ b_{i+1} - b_i }{h_i} \\
      c_i =& \dfrac{ Y_i - Y_{i-1} }{ h_{i-1} } - b_{i-1} h_{i-1}  - a_{i-1} h_{i-1} ^2
    \end{aligned}
  \end{equation}


\end{frame}
\mode*
\mode<all>

