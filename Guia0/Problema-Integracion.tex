\section{Integración Numérica}

\mode<article>

Este problema esta incluido principalmente por razones históricas, 
pero sirve para ilustrar algunos conceptos. Como se ha visto en
la teórica, la integral de una función $f(x)$ en el intervalo 
$[a, b]$ puede aproximarse por distintos métodos, resumidos en la
\autoref{FiguraResumenMetodosIntegracion}. 

Los métodos de Trapecios y Simpson dividen el intervalo de integración 
en $N$ intervalos de igual tamaño. \emph{Trapecios} aproxima el 
área bajo la curva por un trapecio, mientras que \emph{Simpson}
aproxima la integral en dos intervalos consecutivos como la 
integral bajo un polinomio que pasa por los los puntos 
definidos por la función. De ahí surge que para aplicar
este método es necesario definir un número de intervalos con
una paridad definida. 

El método más novedoso para nosotros es el de cuadraturas de gauss.
En lugar de definir un intervalos regulares para subdividir el 
intervalo de integración, la idea es definir un conjunto de puntos
$t_i$ con $i \in [ 1 , N ]$ elegidos de manera que para cualquier 
polinomio $p_N (t)$ de grado $N$  se tiene la mejor aproximación a 
la integral 
\begin{equation}
 \int_a^b  p_N(t) dt = \sum_i^N W_i p_N (t_i) 
\end{equation}

\begin{figure}
  
  \includeslide[width=\textwidth]{FrameResumenMetodosIntegracion}
  \caption{\protect\label{FiguraResumenMetodosIntegracion}}

\end{figure}

La condición de minimizar el error conduce a un sistema de ecuaciones sobre los 
factores $ W_i $ que pueden entenderse de la siguiente manera: se han definido 
pesos $W_i$ y puntos $t_i$ que aproximan a cualquier función $f(t)$ minimizando
el error en la integral. Otra ventaja es que una vez resuelto el sistema de 
ecuaciones de la \autoref{FiguraResumenMetodosIntegracion} los 
valores obtenidos valen para cualquier función $f$ . Los 
métodos de Simpson y Trapecios requerían hacer numerosas evaluaciones de la
función. Las cuadraturas de Gauss minimizan también la cantidad de evaluaciones. 

Una última cosa a tener en cuenta, es que el sistema de ecuaciones
para los $W_i, t_i$ es única e independiente de la función a 
integrar. La solución de ese sistema de ecuacionies se alcanza 
con los polinomios de Chebyshev-Gauss-Lobatto y existen 
scripts donde se resuelven paravarios lenguajes.

En la \autoref{FiguraErroresMetodosIntegracion} se observa el comportamiento del
error para los distintos métodos de integración en función del 
número de intervalos , que corresponde con el número de evaluaciones 
de la función.Se ve claramente que las cuadraturas de Gauss 
alcanzan rápidamente errores muy pequeños. 

De hecho, para las cuadraturas de gauss el error satura en el 
épsilon de la máquina cuando para los otros métodos apenas se alcanzó
un error razonablemente bajo. 

\mode*

%\begin{figure}
%  \includeslide[width=\textwidth]{FiguraErroresMetodosIntegracion}
%  \caption{\label{FiguraErroresMetodosIntegracion}}
%\end{figure}

\begin{frame}<presentation>[label=FrameResumenMetodosIntegracion]
  \frametitle{Métodos de Integración}
  \center
  \includegraphics[width=0.8\textwidth,page=9,trim=0cm 0cm 0cm 4.2cm,clip]{./Resumen/GUIA1-Resumen-2018.pdf}
\end{frame}

\begin{frame}<presentation>[label=FrameResumenMetodosIntegracion]
  \frametitle{Métodos de Integración}
  \center
  \includegraphics[height=0.9\textheight]{./Guia0-Matlab/Ejercicio2/plot-errores.pdf}
\end{frame}

\mode<all>
