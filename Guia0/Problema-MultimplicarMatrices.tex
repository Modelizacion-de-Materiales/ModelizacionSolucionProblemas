\section{Multiplicación de Matrices}

\mode<article>

Este problema se incluye solamente como 
ejercicio de memoria. La multiplicación de matrices
y en particular el concepto de la multiplicación 
de una matriz por un vector columna será uno de 
los cimientos de nuestra materia. 

Recordemos simplemente, 

\mode*
\begin{frame}[label=FrameEquationMatMul]
  \frametitle<presentation>{Multiplicación Matriz $\times$ Vector Columna}

  \begin{equation}
    \begin{pmatrix}
      a_{1,1} & a_{1,2} & \dots & a_{1,n}  \\
	   a_{2,1} & a_{2,2} & \dots & a_{2,n} \\
	   \hdotsfor{4} \\
	   a_{n,1} & a_{n,2} & \dots & a_{n,n} 
    \end{pmatrix}
    \begin{pmatrix}
      x_{1} \\ x_{2} \\ \vdots \\ x_n 
    \end{pmatrix}
  =
    \begin{pmatrix}
      a_{1,1} x_1 +  a_{1,2} x_2 +  a_{1,n} x_n  \\
      a_{2,1} x_1 +  a_{2,2} x_2 +  a_{2,n} x_n  \\
      \vdots \\
      a_{n,1} x_1 +  a_{n,2} x_2 +  a_{n,n} x_n 
    \end{pmatrix}
  \end{equation}
%
%
%
%    =

%    \begin{pmatrix}
%      a_{1,1} \x_1 +  a_{1,2} \x_2 & \cdots & a_{1,n} x_n  \\
%      a_{2,1} \x_1 +  a_{2,2} \x_2 & \cdots & a_{2,n} x_n  \\
%      \vdots \\
%      a_{n,1} \x_1 +  a_{n,2} \x_2 & \cdots & a_{n,n} x_n 
%    \end{pmatrix}


\end{frame}

\mode<all>
