\mode*

\begin{frame}<presentation>[label=FrameModeloIsing]
  \frametitle{Modelo de Ising}

  se trata de un modelo de las interacciones magnéticas entre átomos. 
    $$ E = \frac{J}{2}  \sum _i s_i \sum _{j=1} ^z  s_j $$
  donde 
  \begin{itemize}
     \item $s_i =  \pm 1$
     \item $J$ es la constante de interacción. digamos $J=1$.
     \item $z$ es el numero de primeros vecinos. 
     \item solo consideramos interacción a primeros vecinos.
  \end{itemize}
\end{frame}

\begin{frame}<presentation>[label=FrameRed2D]
  \frametitle{Red 2D de átomos}
  \centering \includegraphics[height=0.7\textheight]{DRAWINGS/RED2d.pdf}
\end{frame}

\begin{frame}<presentation>[label=FrameProblema]
  \frametitle{Problema a resolver}
  \begin{itemize}
      \item<+-> predecir la evolución de un sistema de muchos átomos implica conocer valores medios del
	conjunto
      \item<+-> $ \displaystyle \langle A \rangle =
	  \frac{ \displaystyle \sum _i A_i \exp\bigl(-E_i /kT\bigr) }
	  {\displaystyle \sum_i \exp \bigl( -E_i / kT \bigr) } $
	  , implica conocer todos los estados posibles !
  \end{itemize}

\end{frame}

\mode<all>
