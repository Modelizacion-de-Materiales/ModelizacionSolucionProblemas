\mode*

\begin{frame}<presentation>{label=FrameMétodoMontecarlo}
  \frametitle{Método Montecarlo - Metrópolis}
  \begin{itemize}
     \item La idea es generar una evolución de la distribución en fomra azarosa
     y hacer estadística sobre la evolución de los obserbables.
     \item La evolución de la distribución se permite con una probabilidad de transición.
     \item La Probabilidad de transición entre dos estados se define de manera \emph{ ad-hoc}
     \item 
       $P( W1 \longrightarrow W2 ) = 
       \left\{ 
       \begin{array}{ll}
	 1 & \Delta E _{ W1 \rightarrow W2 } <0 \\
	 \exp \{- \beta \Delta E_{ W1 \rightarrow W2 }\} & \Delta E _{ W1 \rightarrow W2 } >0
       \end{array}
       \right.
       $
      
  \end{itemize}
\end{frame}

\begin{frame}<presentation>[label=FrameMonteCarloAlgorithm]
  \frametitle{Algoritmo}
   
  \includegraphics[width=\textwidth]{DRAWINGS/MC-Algorithm.pdf}

\end{frame}
\mode<all>
