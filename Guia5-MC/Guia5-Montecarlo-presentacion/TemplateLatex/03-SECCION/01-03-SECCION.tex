\begin{frame}{Diapositivas con slides.}

  \begin{itemize}
    \item<1-> Individualizar el problema físico a estudiar: \\ \color{red} lo importante a tener en cuenta son las escalas tanto de tiempo como de distancias involucradas. \par 

      \item<2-> Desarrollar una \color{green} teoría y un modelo matemático  \color{black} para describir el fenómeno a estudiar.
      \item<3-> Llevar el modelo matemático a una forma manejable 
      \item<4-> por un programa de computación.
      \item<5-> Desarrollar o aplicar un algoritmo numérico para tratar 
      \item<6-> el modelo, compatible con su simulación.
      \item<7-> Escribir un código computacional que lo resuelva.
      \item<8-> Realizar el experimento computacional.
      \item<9-> Analizar los datos (mediante gráficos, tablas, curva, etc.).
      \item<10-> Iterar si hiciera falta! \color{green} (muchas veces!) \par  
  \end{itemize}
	\onslide<10-> \color{red} A menudo implica cambiar la forma de pensar un problema!!
\end{frame}
