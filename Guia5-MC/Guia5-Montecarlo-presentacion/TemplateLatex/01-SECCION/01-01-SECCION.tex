%%% Mariano Daniel Forti para
%%% Modelizacion de materiales 
%%% 2019 (c)
% EN estos archivos se carga la info 
% este es un ejemplo de una diapositiva 
% en dos columnas. 
\begin{frame}{Columnas.}

  % texto introductorio
  \centering
  \textbf{  En esta diapositiva 
  se cargan cosas en   dos columnas. }
  \par
 
\begin{columns}
  \column{0.3\textwidth} 
    Primera columna del texto para escribir
    lo que sea necesario.

    puede haber varios parrafos. 


  \column{0.6\textwidth} 
  % el contenido de la columna resaltado
  \begin{beamercolorbox}
    [ht=20pt,wd=\textwidth,center,
    colsep=5pt]
    {highlight1}
\centering Subcolumnas. \par 
  \end{beamercolorbox}
  \begin{columns}
    \column{0.5\textwidth}
      Temas a tratar

    \column{0.5\textwidth}
      \begin{itemize}
	  \item interpolacion
	  \item integracion
	  \item ecuaciones diferenciales
	    ordinarias. 
      \end{itemize}
  \end{columns}
\end{columns}
\end{frame}
