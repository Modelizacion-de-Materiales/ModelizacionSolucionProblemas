\begin{frame}{Una diapositiva con tikz}
\centering
%    \tikz[baseline] \node[anchor=base] (points) {\includegraphics[height=0.4\textheight]{./04-AJUSTE/FIG-DOTS.png}};
%    \tikz[baseline] \node[anchor=base] (interp) {\includegraphics[height=0.4\textheight]{./04-AJUSTE/FIG_SPLINE.png} };
%    \tikz \draw[->,thick,draw=blue] (points.west) -- (interp.east);
  \begin{tikzpicture}
    \node (points) at (-3,0) {\includegraphics[height=0.4\textheight]{./06-SECCION/FIG-DOTS.png}};
    \node (interp) at (3,0) {\includegraphics[height=0.4\textheight]{./06-SECCION/FIG_SPLINE.png} };
    \draw [->,>=latex,fill=blue!40!black,thick,draw=blue,line width=5pt] (points) -- (interp);
  \end{tikzpicture}
\vspace{1cm}
  \note con el paquete tikz es posible hacer graficas en la diapositiva, siendo posible evitar el uso de 
  imagenes externas. 
  \note problema es que cosas tan sencillas como la flecha pueden ser algo complicado de hacer. 
%    \tikz[baseline] \node[anchor=base] (points1) {\includegraphics[height=0.4\textheight]{./04-AJUSTE/FIG-DOTS.png}};
%    \tikz[baseline] \node[anchor=base] (interp1) {\includegraphics[height=0.4\textheight]{./04-AJUSTE/FIG_SPLINE.png} };
%    \begin{tikzpicture}[overlay,>=latex]
%      \path[blue,->] (points1) -- (interp);
%    \end{tikzpicture}

Existen muchos tipos de formas funcionales útiles para interpolar:
\begin{center}
\begin{itemize}
\item Polinomios
\item Funciones Trigonométricas
\item Exponenciales
\end{itemize}
\end{center}
\end{frame}
