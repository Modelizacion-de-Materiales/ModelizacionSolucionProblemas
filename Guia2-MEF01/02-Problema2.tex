\mode<article>
Para resolver el problema 2 debemos refererirnos a lo dicho para el
ensamble de matrices.

\mode*
\begin{frame}[label=FramePuente1]
    \frametitle<presentation>{Problema 2}
    \begin{columns}
        \column{0.5\textwidth}
	\includegraphics[width=\textwidth, page=1]{./PuenteFigure.pdf} / %Ejercicio2/pane-Puente.pdf}
        \column{0.5\textwidth}
            \lstinputlisting{./Puente.ge.show}
    \end{columns}
\end{frame}

\begin{frame}[label=FramePuenteMatrices]
    \frametitle<presentation>{Problema 2: Matrices Elementales}
    \begin{columns}
        \column{0.4\textwidth}
            \lstinputlisting{./MatricesShow1}
        \column{0.6\textwidth}
        \begin{codeblock}
        \verbatiminput{./MatricesShow2}
        \end{codeblock}
    \end{columns}
\end{frame}

\begin{frame}[label=FramePuenteSolucion]
    \frametitle<presentation>{Problema 2: Puente}
    \begin{columns}
        \column{0.3\textwidth}
            \includegraphics[width=\textwidth,page=2]{./Guia2-Python/Ejercicio2/pane-Puente.pdf}
        \column{0.3\textwidth}
        Desplazamientos
            \lstinputlisting{./PuenteDisplaceShow.dat}
        \column{0.3\textwidth}
        Fuerzas

            \lstinputlisting{./Guia2-Python/Ejercicio2/PuenteForces.dat}
    \end{columns}
\end{frame}
\mode<all>
