\mode<article>
Para resolver el problema 2 debemos refererirnos a lo dicho para el
ensamble de matrices.

\mode*
\begin{frame}<presentation>[label=FramePuente1]
    \frametitle{Problema 2}
    \begin{columns}
        \column{0.5\textwidth}
        \includegraphics[width=\textwidth, page=1]{./Guia2-Python/Ejercicio2/pane-Puente.pdf}
        \column{0.5\textwidth}
        \begin{codeblock}
            %\verbatiminput{./Guia2-Python/Ejercicio2/Puente.ge.show}
            \verbatiminput{./Puente.ge.show}
        \end{codeblock}
    \end{columns}
\end{frame}

\begin{frame}<presentation>[label=FramePuenteMatrices]
    \frametitle{Problema 2: Matrices Elementales}
    \begin{columns}
        \column{0.4\textwidth}
        \begin{codeblock}
        %\verbatiminput{./Guia2-Python/Ejercicio2/MatricesShow1}
            \verbatiminput{./MatricesShow1}
        \end{codeb6ock}
        \column{0.6\textwidth}
        \begin{codeblock}
        \verbatiminput{./MatricesShow2}
        \end{codeblock}
    \end{columns}
\end{frame}

\begin{frame}<presentation>[label=FramePuenteSolucion]
    \frametitle{Problema 2: Puente}
    \begin{columns}
        \column{0.5\textwidth}
            \includegraphics[width=\textwidth,page=2]{./Guia2-Python/Ejercicio2/pane-Puente.pdf}
        \column{0.25\textwidth}
        Desplazamientos

        \begin{codeblock}
            \verbatiminput{./PuenteDisplaceShow.dat}
        \end{codeblock}
        \column{0.25\textwidth}
        Fuerzas

        \begin{codeblock}
            \verbatiminput{./Guia2-Python/Ejercicio2/PuenteForces.dat}
        \end{codeblock}
    \end{columns}
\end{frame}
\mode<all>
