\mode<article>

Para resolver este problema debemos recordar
que las fuerzas distribuidas pueden tranformarse en 
fuerzas nodales mediante la integral:

\mode*

\begin{frame}<presentation>[label=FrameProblema3Forces]
    \frametitle{Problema 3: Fuerzas Distribuidas}

    \begin{equation}
        F_e = \int _{elemento}{
            \begin{bmatrix} N_1(x) \\ N_2(x) \end{bmatrix}
        f (x)
        dx
    }
    \end{equation}

\end{frame}

\begin{frame}<presentation>[label=FrameCalculoFuerzas]
    \frametitle{Calculo de fuerzas Directo}
    \begin{columns}
        \column{0.4\textwidth}
            \begin{codeblock}
                \verbatiminput{./CalculoFuerzasShow.py}
            \end{codeblock}

        \column{0.6\textwidth}
            \begin{codeblock}
                \verbatiminput{./CalculoFuerzasShow2.py}
            \end{codeblock}

    \end{columns}
\end{frame}
\begin{frame}<presentation>[label=FramePaneDesplaza]
    \frametitle{Desplazamientos}
    \center
    \includegraphics[width=0.6\textwidth,trim=0cm 0cm 22cm 0cm, clip=true]
    {./Guia2-Python/Ejercicio3/pane-results.pdf}

\end{frame}
\begin{frame}<presentation>[label=FramePaneTensiones]
    \frametitle{Tensiones}
    \begin{columns}
        \column{0.1\textwidth}
        $$ \tau = \frac{\Delta l_e}{l_e} E$$
        \column{0.8\textwidth}

    \center
    \includegraphics[width=0.8\textwidth,trim=22cm 0cm 0cm 0cm, clip=true]
    {./Guia2-Python/Ejercicio3/pane-results.pdf}


    \end{columns}
\end{frame}
