%%% BEAMER THEME
% el tema 'boxes' es igual al default pero permite definir boxes de estructura 
% a mano. 
\mode<presentation>{
  \usetheme{boxes}
  % los boxes que identifican lo que se esta leyendo
    % box de la izquierda: la materia (subtitulo)
    \addheadbox{structure2}{\quad \tiny \insertshortsubtitle}
  %  box del medio en cabecera, el titulo de la clase
    \addheadbox{structure0}{\quad \tiny  \inserttitle \quad } 
  % box en a la derecha , eltítulo de la sección. 
    \addheadbox{structure1}{\quad \tiny \insertsection}
}
% tema interno y de colores para las diapositivas normales. 
\useinnertheme{rectangles}
\usecolortheme{dove}
% la fuente de las ecuaciones
\usefonttheme[onlymath]{serif}

% entorno codeblock para meter piezas de código.
% el color se definió en BEAMERCOLORS

\newenvironment{codeblock}
{
  \begin{beamercolorbox}{codeblock}
    \usebeamerfont{codeblock}
}
{
  \end{beamercolorbox}
}

