\subsection{recuadros en el texto}


\begin{frame}{Recuadros simples alrededor del texto: tcolorbox}

  \begin{itemize}
  %% 
  \note{ \textbf{tcolorbox} parece ser la opcion más simple para estos casos }
  \note{ el comando \emph{ \\pause } genera slides! }
 \tcbset{colframe=blue,colback=white,width=\textwidth}
 \begin{tcolorbox}
      \item  Individualizar el problema físico a estudiar: \\ \color{red} lo importante a tener en cuenta son las escalas tanto de tiempo como de distancias involucradas. 
 \end{tcolorbox}
 \pause
      \item  Desarrollar una \color{green} teoría y un modelo matemático  \color{black} para describir el fenómeno a estudiar.
      \item  Llevar el modelo matemático a una forma manejable 
      \item  por un programa de computación.
      \item  Desarrollar o aplicar un algoritmo numérico para tratar 
      \item  el modelo, compatible con su simulación.
      \item  Escribir un código computacional que lo resuelva.
  \pause
  \note{ la posición de los tcolorbox se puede modificar asi: }
  \begin{tcolorbox}[enlarge left by=-0.1\textwidth , width=1.1\textwidth ]
      \item  Realizar el experimento computacional.
      \item  Analizar los datos (mediante gráficos, tablas, curva, etc.).
      \item   Iterar si hiciera falta! \color{green} (muchas veces!) \par  
  \end{tcolorbox}
  \pause
  \end{itemize}
  \color{red} A menudo implica cambiar la forma de pensar un problema!!

\end{frame}
